% Feb 26 2016
% based on Jobs/resume/cv.tex and Jason's resume.tex


\begin{small}

\begin{center}
525 Davey Lab\\
University Park, PA 16802\\
(814) 321-7236\\
wang.xuesong.sharon@gmail.com\\
bit.ly/sharonxuesongwang\\


\end{center}
\smallskip



%%%%%%%%%%%%%%%%%%%%%%%%%%%%%%%%%%%%%%%%%%%%%%%%%%%%%%%%%%%%%%%%%%%%%%%%%%%%%%%%%%%%%%%%%%%%%%%%
\topic{\centerline{EDUCATION}}

{\sl PhD Candidate}, Astronomy \& Astrophysics, Penn State University
\hfill expected Aug 2016 \\
{\sl PhD Minor}, Computational Sciences\\
Thesis: \textit{Finding More and Lower Mass Exoplanets with Improved
  Radial Velocimetry}\\ 
Advisor: Dr.\ Jason T.\ Wright
\vspace{10pt}

{\sl Bachelor of Science}, Physics, Tsinghua University, Beijing,
China \hfill Jun 2008 \\
Thesis: \textit{Characterizing the Luminosity-Variability Correlation
  in Gamma-Ray Bursts}\\
Advisor: Dr.\ Shuang-Nan Zhang



%%%%%%%%%%%%%%%%%%%%%%%%%%%%%%%%%%%%%%%%%%%%%%%%%%%%%%%%%%%%%%%%%%%%%%%%%%%%%%%%%%%%%%%%%%%%%%%%

\topic{\centerline{PROFESSIONAL EMPLOYMENT}} 


{\sl Research Assistant / Fellow} \hfill        May 2010 -- present  \\
Department of Astronomy \& Astrophysics, Penn State University

\vspace{10pt}   
   
{\sl Teaching Assistant} \hfill       Aug 2008 -- Dec 2009 \\
Department of Astronomy \& Astrophysics, Penn State University
   

%%%%%%%%%%%%%%%%%%%%%%%%%%%%%%%%%%%%%%%%%%%%%%%%%%%%%%%%%%%%%%%%%%%%%%%%%%%%%%%%%%%%%%%%%%%%%%%%

\topic{\centerline{AWARDS}} 


{\sl Carnegie DTM Fellowship in Astronomy and Planetary Science} \hfill     since Aug 2016 
\vspace{10pt}

{\sl NASA Earth and Space Science Fellow} \hfill     since Sep 2014 \\
Proposal title: Finding the Lowest Mass Exoplanets with Improved
Radial Velocimetry
\vspace{10pt}

{\sl Downsborough Graduate Fellowship} \hfill        May 2013 \\
{\sl Stephen B. Brumbach Fellowship in Astrophysics} \hfill        May 2010 \\
{\sl Zaccheus Daniel Fellowship} \hfill        2009-2013 \\
{\sl Teaching Assistant of the Year Award} \hfill        Jun 2009 \\
Department of Astronomy \& Astrophysics, Penn State University

   
%%%%%%%%%%%%%%%%%%%%%%%%%%%%%%%%%%%%%%%%%%%%%%%%%%%%%%%%%%%%%%%%%%%%%%%%%%%%%%%%%%%%%%%%%%%%%%%%
\newpage
\topic{\centerline{TELESCOPE TIME AWARDED AND OBSERVING EXPERIENCE}} 


{\bf Exoplanet Programs}

PI, 25.7 hours on Hobby-Eberly Telescope, with the High Resolution Spectrograph \hfill 2013 \\
{\it Improve the Radial Velocity Precision of HET/HRS}\\
Co-I: Jason Wright, Ming Zhao
\vspace{10pt}

Observer, Observing Planner, Tull Spectrograph at the McDonald Obs.\ 2.7m
Telescope \hfill 2013\\
TS12 arm, R$\sim$500,000, day-time runs
\vspace{10pt}

Observer, Keck/HIRES remote observing at Caltech and Yale ROCs \hfill 2010, 2011, 2013\\
\vspace{10pt}

{\bf Extragalactic Programs}

As founding member of the MUSSCEL program (MUltiwavelength Study of the
Structure, Chemistry and Evolution of LSB galaxies):
\vspace{10pt}

Co-I, 5 hours of Green Bank Telescope, 2015A with AUGUS receiver
\hfill 2014 \\
{\it CO in Low Surface Brightness Galaxies in Tandem with Optical/UV Star Formation}\\
PI: Jason Young, Co-Is: Rachel Kuzio de Naray, Karen O'Neil
\vspace{10pt}

Co-I, 9 Nights on VIRUS-P IFU on 2.7m telescope of McDonald
Observatory \hfill 2013, 2014, 2015 \\
{\it IFU Spectroscopy of Low Surface Brightness Galaxies} \\
PI: Jason Young, Co-I for 2014 \& 2015: Rachel Kuzio de Naray
\vspace{10pt}

Co-I, NASA Swift Cycle 10 GI Program \hfill 2013 \\
{\it Anchoring the Blue End of Low Surface Brightness Disk Galaxy SEDs}\\
PI: Jason Young, Co-I: Rachel Kuzio de Naray
\vspace{10pt}

Others: Co-I on one {\it Fermi} proposal on GRB theory (2009) and one {\it
  Chandra} Archival proposal on AGN spectroscopy (2013).


%%%%%%%%%%%%%%%%%%%%%%%%%%%%%%%%%%%%%%%%%%%%%%%%%%%%%%%%%%%%%%%%%%%%%%%%%%%%%%%%%%%%%%%%%%%%%%%%

\topic{\centerline{TALKS AND CONFERENCE POSTERS}} 


{\bf Talks}
\vspace{10pt}

{\sl Paths, Roadblocks, and Byways in Detecting Habitable Rocky
  Planets in Radial Velocity Data} \\
Invited Talk, Carnegie DTM Exoplanet Seminar \hfill Nov 2015\\
Invited Talk, Berkeley Center for Integrative Planetary Science Seminar \hfill Sep 2015\\
NExScI Exoplanet Seminar \hfill Sep 2015\\
Contributed Talk, Bay Area Exoplanet Science Meeting \hfill Sep 2015

\vspace{10pt}
{\sl Co-Chair, Breakout Discussion Session on Telluric Contamination} \hfill
Jul 2015 \\
The 2nd Extremely Precise Radial Velocity Workshop, Yale

\vspace{10pt}
{\sl Improve RV Precision through Better Modeling and
  Better Reference Spectra} \hfill  May 2015\\
Contributing Talk, The 1st Emerging Researchers in Exoplanet
Symposium, Penn State

\vspace{10pt}
{\sl Pushing the Radial Velocity Precision to 1~m/s} \hfill   Oct 2014 \\
Stellar, Solar and Planet Seminar, Harvard/CfA
% 

\vspace{10pt}
{\sl Accreting Supermassive Black Holes in Submm Galaxies} \hfill   Apr 2013 \\
Contributed Talk at the Penn State Neighborhood Cosmology Workshop
% 

\vspace{10pt}
{\sl AGNs in Submm Galaxies --- Combining the Power of {\it Chandra}
  and ALMA}   \\
Contributed Talk at 2013 AAS Winter Meeting, Long Beach \hfill Jan
2013 \\
Contributed Talk at Seyfert 2012 Workshop --- Nuclei of Seyfert Galaxies and QSOs \\
Max Planck Institute for Radio Astronomy, Bonn, Germany \hfill Nov 2012
%\vspace{-15pt} 

\vspace{10pt}
{\sl Resolving the 6-8 keV X-ray Background} \hfill   Aug 2012 \\
Lunch Talk at Kavli Institute of Astronomy \& Astrophysics \\
Peking University, Beijing, China
%

\vspace{10pt}
plus 6 Penn State Department of Astronomy \& Astrophysics Lunch Talks
and 2 invited talks at the {\it Swift} Mission Control Center.

\vspace{10pt}
{\bf Posters}

\vspace{10pt}
{\sl Telluric Contamination: Effects and Solutions} \hfill Jul 2015 \\
Poster at the 2nd Extremely Precise Radial Velocity Workshop, Yale

\vspace{10pt}
{\sl Finding Extra-Solar Planet Near and Far} \hfill   Mar 2013 \\
Poster Presentation at the 2013 Penn State Graduate Exhibition \\
{\bf First Prize Winner} in the Physical Sciences and Mathematics
Category
% 

\vspace{10pt}
{\sl Improving the Radial Velocity Precision of HET/HRS} \hfill
May 2011, Jan 2014 \\
Serial Poster Presentations at the 2011 AAS Winter Meeting in Seattle
and Summer Meeting in Boston, the 1st Precise Radial Velocity Workshop
at Penn State, and the 2014 AAS Winter Meeting in National
Harbor.
%

\vspace{10pt}
{\sl Spectral Lags from Structured Jets}  \\
Poster Presentation at the 2010 AAS Winter Meeting in D.C \hfill Jan
2010 \\
Poster Presentation at the Swift 5 Year Conference, {\bf Poster Award
  Winner} \hfill Nov 2009
%\vspace{-15pt} 


%%%%%%%%%%%%%%%%%%%%%%%%%%%%%%%%%%%%%%%%%%%%%%%%%%%%%%%%%%%%%%%%%%%%%%%%%%%%%%%%%%%%%%%%%%%%%%%%
 
\topic{\centerline{SUMMER SCHOOLS AND TRAININGS}} 


{\sl The Dunlap Institute Summer School on Astronomical
  Instrumentation} \hfill Aug 2013 \\
Honorable Mention, Optical Design Challenge

\vspace{10pt}
{\sl The AAS CAE Tier I Workshop on Teaching Astro 101} \hfill 2011 

\vspace{10pt}
{\sl The Summer School in Statistics for Astronomers} \hfill  Jul 2010\\
Pennsylvania State University
% 

\vspace{10pt}
{\sl The 37th Stanford SLAC Summer School} \hfill        Aug 2009  \\
Revolutions on the Horizon: A Decade of New Experiments\\
Honorable Mention, The 37th SLAC Summer School Challenge
% 


%%%%%%%%%%%%%%%%%%%%%%%%%%%%%%%%%%%%%%%%%%%%%%%%%%%%%%%%%%%%%%%%%%%%%%%%%%%%%%%%%%%%%%%%%%%%%%%%
\newpage
\topic{\centerline{SERVICES AND COMMITTEE WORK}} 


{\sl Referee, ApJ, A\&A} 
\vspace{10pt}

{\sl Outreach Volunteer} \hfill  since Aug 2008 \\
Given over 10 planetarium shows to local school students and general public, and over \\
6 public talks at various outreach events through the Penn State Astro
Outreach program. 
\vspace{10pt}

{\sl Astronomy beyond Academia, Founder and Group Manager} \hfill  since Aug 2012 \\
A {\it LinkedIn} network for astronomers outside academia,
endorsed by AAS Employment Committee
\vspace{10pt}

{\sl Mentor for First-Year Physics Major Undergraduate} \hfill since Sep 2014 \\
Penn State Physics and Astronomy Women Mentoring Program
\vspace{10pt}

{\sl Scientific and Logistic Organizing Committee Member} \hfill May
2015 \\
The 1st Emerging Researchers in Exoplanet Symposium, Penn State
\vspace{10pt}

{\sl Graduate Council Representative} \hfill        Sep 2010 -- May 2012  \\
Penn State Graduate Student Association 
\vspace{10pt}

 {\sl Co-Chair and Event Organizer for Inside Scientists Studio} \hfill        Sep 2010 -- May 2011  \\
Graduate Women in Science, Nu Chapter at Penn State 
\vspace{10pt}

{\sl Mentor for First-Year International Graduate Students} \hfill
2009 -- 2010 \\
Penn State Global Programs
\vspace{10pt}


%%%%%%%%%%%%%%%%%%%%%%%%%%%%%%%%%%%%%%%%%%%%%%%%%%%%%%%%%%%%%%%%%%%%%%%%%%%%%%%%%%%%%%%%%%%%%%%%
 
\topic{\centerline{TECHNICAL SKILLS}} 


{\bf Coding Languages:} \\
IDL, Python, Java, C$++$, R
\vspace{10pt}

{\bf Astronomical Data Analysis Skills:}\\
Exoplanet:
\begin{itemize}
\item Forward modeling echelle spectra for radial velocity (RV) extraction;
  \begin{itemize}
  \item working with and improving the California
    Planet Survey Doppler code (used at Keck/HIRES, APF, HET/HRS, AAT,
    etc.)
  \item building a Doppler code from scratch
  \end{itemize}
\item Diagnosing and solving problems in the context of
  iodine precise RV;
  \begin{itemize}
  \item general diagnostic tests with calibration frames, standard stars
    frames, etc.;
  \item modeling telluric contamination in reference and science spectra;
  \item modeling spectrograph response function (spectral PSF);
  \item modeling/characterizing iodine atlases (as calibration/reference spectra);
  \end{itemize}
\item Observation and raw data reduction with echelle spectrograph;
\item Characterization of planetary systems with RV data;  
\item Modeling telluric absorption lines;
\item Optical and NIR photometry;
\item Solid background in statistical computing.
\end{itemize}

Extragalactic:
\begin{itemize}
\item X-ray: photometry, stacking, spectroscopy, and spectral modeling (CIAO
tools and XSPEC) 
\item Galaxy Stellar SED fitting (UV, optical to NIR; experience with FAST, GalMC, and CIGAR)
\item Metalicity estimate from emission lines (e.g.\ using the R23 method)  
\end{itemize}  

{\bf Astronomical Packages and Software:} \\
California Planet Survey Consortium Doppler Code\\
REDUCE (optimal extraction for 2-D echelle spectrum)\\
TERRASPEC (software for modeling telluric spectra based on HITRAN line
database)\\
IodineSpec5 (theoretical computation of iodine lines)\\
SourceExtractor (Optical/NIR photometry) \\
CIAO tools and XSPEC (X-ray photometry and spectroscopy)\\
FAST (galaxy SED fitting)\\
ALMA Observing and Proposal Tool
\vspace{10pt}

{\bf Published Code:} \\
BOOTTRAN (in IDL, bootstrapping to compute
error bars for Keplerian orbit parameters, including transit
ephemeris, based on radial velocity data)


%%%%%%%%%%%%%%%%%%%%%%%%%%%%%%%%%%%%%%%%%%%%%%%%%%%%%%%%%%%%%%%%%%%%%%%%%%%%%%%%%%%%%%%%%%%%%%%%
\topic{\centerline{LIST OF PUBLICATIONS}}


Total publications: 13, with 4 as first or second author, 9 as
contributing author.\\ Total citations: 225 (152 citations as first or
second author), h-index: 9, as of Mar. 2016. \\
1 first author paper and 2 co-author papers in preparation.

\vspace{10pt}
{\bf Publications as a Major Contributor:}
\vspace{10pt}

4. The Exoplanet Orbit Database II: Updates to exoplanets.org \\
Eunkyu Han$^+$, {\bf Sharon X. Wang}, Jason T. Wright, et al. 2014, {\it
  PASP}, 126, 813\\
($^+$ Undergraduate student co-supervised)


3. The X-ray Properties of the Submillimeter Galaxies in the ALMA \\
LABOCA E-CDF-S Submillimeter Survey \\
{\bf Sharon Xuesong Wang}, W.\ Niel Brandt, et al. 2013, {\it ApJ}, 778, 179


2. The Discovery of HD 37605$c$ and A Null Detection of Transits of HD
37605$b$ \\
{\bf Sharon Xuesong Wang}, Jason T. Wright, et al. 2012, {\it ApJ}, 761, 46


1. Tracking Down the Source Population Responsible for the Unresolved Cosmic 6-8 keV Background \\
Yongquan Xue, {\bf S. X. Wang}, et al. 2012, {\it ApJ}, 758, 129

\vspace{10pt}
{\bf Other Publications:}


9. The Distribution of Star Formation and Metals in the Low Surface
Brightness Galaxy UGC 628 \\
Young, J.\ E.; Kuzio de Naray, Rachel; \textbf{Wang, Sharon X.}, 2015,
{\it MNRAS}, 452, 2973


8. Evolution in the Black Hole---Galaxy Scaling Relations and the Duty
Cycle of Nuclear Activity in Star-forming Galaxies \\
Mouyuan Sun, and other 8 coauthors including Sharon X.\ Wang,
2015, {\it ApJ}, 802, 14S


7. The California Planet Survey IV: A Planet Orbiting the Giant Star
HD 145934 and Updates to 7 Systems with Long-Period Planets\\ 
Katherina Y. Feng, Jason T. Wright, Ben Nelson, \textbf{Sharon
  X. Wang}, et al.
2014, {\it ApJ}, 800, 22F


6. MARVELS-1: A Face-on Double-lined Binary Star Masquerading as a
Resonant Planetary System and Consideration of Rare False Positives in
Radial Velocity Planet Searches \\
Jason T.~Wright, Arpita Roy, Suvrath Mahadevan, \textbf{Sharon X.~Wang}, et al.
2013, {\it ApJ}, 770, 119


5. Host Star Properties and Transit Exclusion for the HD 38529
Planetary System \\
Gregory W.\ Henry, Stephen R.\ Kane, {\bf Sharon X.\ Wang}, et al. 2013, {\it ApJ}, 768, 155


4. The HD 192263 System: Planetary Orbital Period and Stellar Variability Disentangled \\
Diana Dragomir, and other 13 coauthors including Sharon X. Wang, 2012, {\it ApJ}, 754, 37


3. A Search for the Transit of HD 168443b: Improved Orbital Parameters and Photometry \\
Genady Pilyavsky, and other 15 coauthors including Sharon X. Wang, 2011, {\it ApJ}, 743, 162


2. Stellar Variability of the Exoplanet Hosting Star HD 63454 \\ 
Stephen R.\ Kane, and other 12 coauthors including Sharon X. Wang, 2011, {\it ApJ}, 737,  58


1. Revised Orbit and Transit Exclusion for HD 114762b \\
Stephen R.\ Kane, and other 6 coauthors including Sharon X. Wang, 2011, {\it ApJ}, 735,  L41



\end{small}
