\chapter{Introduction}

%%%%%%%%%%%%%%%%%%%%%%%%%%%%%%%%%%%%%%%%%%%%%%%%%%%%%%%%%%%%%%%%%%%%%%%%%%%%%%
\section{On Detecting New Worlds}
The detection of extra-solar planets (exoplanets). What was conjured.
What was claimed early on and disputed. The first pulsar planets using timing.

The first exoplanets around main-sequence stars were discovered by the
radial velocity (RV) method, where precise Doppler spectroscopy
measures the wavelength shift of the host stars induced by the
gravitational pull of the planets \citep{1988ApJ...331..902C,
  1989Natur.339...38L, 1993ApJ...413..339H, 1995Natur.378..355M,
  1996ApJ...464L.153B}. Since then, the RV method has discovered
hundreds of planetary systems (see exoplanets.org; \citealt{eod2014})
and contributed to numerous confirmation and characterization of
exoplanets discovered by the transit method (e.g., for
\kepler\ follow-up observations; \citealt{Marcy2014}).

Later on, transit detections and micro-lensing detections. See review
by Wright and Gaudi for more.

What is the landscape for planet detection today and in the near
future. How RV fits in. 

%%%%%%%%%%%%%%%%%%%%%%%%%%%%%%%%%%%%%%%%%%%%%%%%%%%%%%%%%%%%%%%%%%%%%%%%%%%%%%
\section{The Art of Precise Doppler Spectroscopy}

History of precise Doppler spectroscopy. Early work by Campbell and
Walker. First detections with ThAr. Quickly progressed to 3m/s with
iodine. HARPS revoluntionized the field by showing what a cryogenic 
spectrometer can do.

What is the landscape of RV instruments today.

The current best RV precision is around 1~m/s \citep{eprv2015},
attainable via two wavelength calibration methods in the optical band:
ThAr lamp emission line calibration (e.g., ELODIE and HARPS;
\citealt{elodie, harps-s}; $\sim$400-690~nm) and iodine cell
absorption line calibration (e.g., Keck/HIRES and Magellan/PFS;
\citealt{butler1996, 2010SPIE.7735E..53C}; $\sim$500-620~nm). The
major obstacles for achieving a higher RV precision are: stellar
activity induced RV signals, instrumental effects, telluric
contamination, and limitation in data analysis \citep{eprv2015}.

Where is the field heading towards in the near future. Fischer et
al. review, Plavchan NASA white paper. Future instruments.

%%%%%%%%%%%%%%%%%%%%%%%%%%%%%%%%%%%%%%%%%%%%%%%%%%%%%%%%%%%%%%%%%%%%%%%%%%%%%%
\section{Precise Doppler Spectroscopy with Iodine Cells as
  Calibrators} 

Much history in Paul Butler's personal account.

Our work mostly deals with \het\ and \keck, who happen to only have
iodine-calibrated precise RV instruments so far. Unfortunately this
will remain true for a while in the near future too for all large
telescopes. Getting the most out of the iodine calibration method is
therefore meaningful. 

Iodine method will remain relevant in the future too,
because iodine is a relatively cheap way for getting precise RVs,
which will be on high demand in the near future once TESS is
launched. Study why and why not HET works will inform us about other
iodine-calibrated instruments and current and future fiber-fed
instruments. Study how we can improve Keck and generalize its methodology
to other telescopes is important.

The scope of this work mostly deals with data analysis, with the hope
to diagnose problems and inform hardware builders, and also to improve
the data analysis technique in order to measure RVs more precisely.

introduction on \het\ and \keck\ will be in their chapters.