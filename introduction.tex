\chapter{Introduction}

The first exoplanets around main-sequence stars were discovered by the
radial velocity (RV) method, where precise Doppler spectroscopy
measures the wavelength shift of the host stars induced by the
gravitational pull of the planets \citep{1988ApJ...331..902C,
  1989Natur.339...38L, 1993ApJ...413..339H, 1995Natur.378..355M,
  1996ApJ...464L.153B}. Since then, the RV method has discovered
hundreds of planetary systems (see exoplanets.org; \citealt{eod2014})
and contributed to numerous confirmation and characterization of
exoplanets discovered by the transit method (e.g., for
\kepler\ follow-up observations; \citealt{Marcy2014}).

The current best RV precision is around 1~m/s \citep{eprv2015},
attainable via two wavelength calibration methods in the optical band:
ThAr lamp emission line calibration (e.g., ELODIE and HARPS;
\citealt{elodie, harps-s}; $\sim$400-690~nm) and iodine cell
absorption line calibration (e.g., Keck/HIRES and Magellan/PFS;
\citealt{butler1996, 2010SPIE.7735E..53C}; $\sim$500-620~nm). The
major obstacles for achieving a higher RV precision are: stellar
activity induced RV signals, instrumental effects, telluric
contamination, and limitation in data analysis \citep{eprv2015}.