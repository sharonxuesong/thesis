\chapter{Introduction}

\begin{quote}
``There are infinite worlds both like and unlike this world of
ours. For the atoms being infinite in number, as was already proven,
...there nowhere exists an obstacle to the infinite number of worlds.''
\end{quote}
\hfill Epicurus ($\sim$341-270 B.C.)

\begin{comment}
\begin{quote}
``This space we declare to be infinite... In it are an infinity of
worlds of the same kind as our own.''
\end{quote}
\hfill Giordano Bruno, {\it On the Inifinite Universe and Worlds} (1584)
\end{comment}

\begin{quote}
``How vast those Orbs must be, and how inconsiderable this Earth, the
Theatre upon which all our mighty Designs, all our Navigations, and
all our Wars are transacted, is when compared to them.''
\end{quote}
\hfill Christiaan Huygens, {\it Cosmotheoros} (1698)

\begin{quote}
``Of all of the topics of study in astronomy, exoplanets hold a
special place in the imagination. More than stars, nebulae, or
galaxies, they are {\it places}, ...''
\end{quote}
\hfill \cite{2006PhDT.........8W}

%%%%%%%%%%%%%%%%%%%%%%%%%%%%%%%%%%%%%%%%%%%%%%%%%%%%%%%%%%%%%%%%%%%%%%%%%%%%%%
\section{On Detecting New Worlds}

Even before human beings realized that other stars are like our Sun,
the existence of other worlds have been speculated by ancient greek
philosophers such as Epicurus and Democritus. In the blooming age of
astronomy in the 1400s and 1600s, early pioneers such as Giordano
Bruno and Christiaan Huygens have also pondered upon the existence of
planets around other stars (extra-solar planets, or
exoplanets).\footnote{In fact, Huygens conducted the first documented
search on exoplanets. For more on the history of exoplanet searches,
see these three websites:\\ The NASA PlanetQuest,
http://www.nasa.gov/externalflash/PQTimeline/; \\ Search for
Exoplanets, http://www.hao.ucar.edu/research/stare/search.html; \\
ESO,
https://www.eso.org/public/outreach/eduoff/cas/cas2004/casreports-2004/rep-228/. }
In modern times, 40 years before the discovery of the first exoplanet,
Otto Struve stated that exoplanets, especially ``super-Jupiters'' on
short orbits, should be detectable via spectroscopy and photometry
\citep{1952Obs....72..199S}.

Unfortunately, the earliest claims of exoplanet detections before
1980s all turned out to be erroneous \citep{1855MNRAS..15..228J,
1969AJ.....74..757V}. These and a later retracted claim of a planet
around a pulsar by \cite{1991Natur.352..311B} made all astronomers
extremely cautious about exoplanet detection claims. In 1988,
Campbell, Walker and Yang announced potential planetary signal from
the star Gamma Cephei, but they were hesitant in calling it a
detection due to limitations of early instruments. Their detection
method was to measure the radial velocity (RV) variation of the star
using precise Doppler spectroscopy, which is described in the next
section and is also the theme of this thesis.  It was not until 2003
that the planet around $\gamma$ Cephei A was confirmed
\citep{2003ApJ...599.1383H}, which made the work of
\cite{1988ApJ...331..902C} the first exoplanet detection. The
detection of HD 114762b (``Latham Planet'';
\citealt{1989Natur.339...38L}) also belongs to the family of first
exoplanet detections, though the planet was thought to be a brown
dwarf at the time due to its large mass. A similar story to $\gamma$
Cephei Ab is the discovery of $\beta$ Gemini b
\citep{1993ApJ...413..339H}, where the existence of the planet was not
confirmed until 2006 \citep{2006A&A...457..335H} because of the strong
activity-induced RV signals of the giant host star.\footnote{See
  Chapter 4 of \cite{2013pss3.book..489W} for a more detailed history
  on these early detections.} 

The more commonly recognized first detection of exoplanets belongs to
\cite{1993ApJ...413..339H}, who reported two planets around the pulsar
PSR B1257$+$12, detected via the pulsar timing method using radio data
(later on it turned out this system hosts one more planet). It was a
surprising detection in many aspects, and these planets remains the
only known planetary system around a pulsar to date (as of May
2016). If exoplanets could exist around exotic stars like pulsars,
then it is only natural to expect them to exist around more
``regular'' main-sequence stars like our Sun.

Finally, in 1995, a team in Geneva announced the first definitive
detection of a planet around a main-sequence star, 51 Peg b
\citep{1995Natur.378..355M}. Their results were quickly confirmed by
other planet hunters such as Geoffrey W.\ Marcy and R.\ Paul Butler,
who quickly caught up with the game \citep{1996ApJ...464L.153B} and
went on to detect more than half of the hundreds of known exoplanets
up until the launch of NASA's \kepler\ mission
\cite{2010Sci...327..977B}. The method adopted by
\cite{1995Natur.378..355M} and \cite{1996ApJ...464L.153B} was again
precise Doppler spectroscopy. Today, there are over 585 exoplanets
discovered by precise Doppler spectroscopy. The discoveries by precise
Doppler spectroscopy that happened beyond this point is briefly
accounted for in the next section. 

Several years later, \cite{2000ApJ...529L..41H} and
\cite{2000ApJ...529L..45C} detected the first exoplanet transiting
event, where the planet moves in between the disk of the star and our
line of sight periodically, leaving signals in the stellar light
curves. This transiting planet, HD
209458b, which was discovered via precise Doppler spectroscopy
first. Nonetheless, this discovery opened up the age of ground-based
transit, where projects such as OGLE, TrES, WASP, XO, and HAT etc.\ added
more than 200 new exoplanet discoveries to date
\citep{2003Natur.421..507K, 2004ApJ...613L.153A, 2006MNRAS.372.1117C,
  2006ApJ...648.1228M, 2007ApJ...656..552B}. 

In 2009, the discovery of exoplanets entered a new era with the launch
of NASA's \kepler\ satellite, which is a dedicated space mission to
detect transiting exoplanets. This extremely fruitful mission has made
new exoplanet discoveries in the counts of thousands
\citep{2014ApJ...784...45R, 2016ApJ...822...86M}, with over 2000 more
planet candidates (see, e.g., NASA Exoplanet Archive for statistics on
exoplanet discoveries). The science of exoplanets expanded from the
philatelic style to including population and statistical studies which
inform planet formation and evolution in powerful ways more than ever
(e.g., \citealt{2013ApJ...766...81F} on occurrence rate and
\citealt{2015arXiv150407557W} on composition distribution). As of May
23 2016, there are 3268 confirmed exoplanets, to which the transit
method contributed 2569 (585 discovered by Doppler spectroscopy).

Besides using precise Doppler spectroscopy and transits, other methods
have also made unique and important discoveries of exoplanets, as they
probe different stellar population and are subject to different
observational biases. \cite{2004ApJ...606L.155B} made the first
micro-lensing detection of exoplanet, where planets acts as additional
gravitational lenses beside their host star and leave characteristic
signatures onto the light curves of the background star. There are 37
exoplanets discovered via micro-lensing so far. Astronomers also
directly detected light from young exoplanets around young stars via
direct imaging, the first of which are Fomalhaut b
\citep{2008Sci...322.1345K} and the four planets around HR 8799
\citep{2008Sci...322.1348M}. Today, there are 41 directly imaged
planets.

More exoplanets around more diverse host stars are expected to be
discovered in the near future, with many new missions and surveys
being carried out, built, or planned. Post-2013, \kepler\ continued as
the K2 mission (\kepler\ on two reaction wheels) and kept churning out
planets (e.g., \citealt{2016ApJS..222...14V}). The Transiting
Exoplanet Survey Satellite (TESS; \citealt{2014SPIE.9143E..20R};
expected to launch in Summer 2017) will survey the whole sky,
targeting nearby and bright stars, including the previously relatively
unexplored population of M dwarf stars. There is no doubt that, once
again, exoplanet discoveries will be made in counts of
thousands. Ongoing surveys with the Gemini Planet Imager on Gemini
South \citep{2014PNAS..11112661M} and the SPHERE instrument on the
Very Large Telescope \citep{2008SPIE.7014E..18B} are populating
exoplanets in a new parameter space (young stellar/planetary age and
moderate to long orbital distances). The future for micro-lensing
discoveries also remains bright as thousands of exoplanets are
expected to be found by NASA's WFIRST-AFTA mission
\citep{2014arXiv1409.2759Y}.

Among all these exciting discoveries happening or on the horizon,
precise Doppler spectroscopy continues to play an important role. It
is the most important method for measuring planetary
masses,\footnote{Planetary masses can also be measured via studies on
the transit timing variations (TTVs) due to the dynamic interactions
of multiple planets (e.g., \citealt{2016ApJ...820...39J}), but TTVs
are only measurable for a small fraction of all transiting planets
\citep{mazeh2013}. \cite{2013Sci...342.1473D} have also developed an
innovative method to estimate planetary mass via transmission
spectroscopy of the planetary atmosphere, but the method is
model-dependent and requires a large amount of large-aperture space
telescope time (e.g., hundreds of orbits of JWST).} and it will remain
a crucial independent method for discovering new exoplanets (after
all, only a small fraction of exoplanets happen to pass in between
their host star and the Earth). The synergy between the \kepler\
mission and the ground-based Doppler spectroscopy follow-ups has
demonstrated the power of this new exoplanet discovery and
characterization ``routine'', where RVs are presented as the
convincing evidence for the planetary nature of the transit signal,
and they also provide valuable information on the planetary masses and
thus their bulk densities (e.g., \citealt{marcy2014}). Such
measurements are crucial for mapping out the demographics of
exoplanets. However, only a small fraction of \kepler\ targets have
been followed up by Doppler spectroscopy, and the future discoveries
of TESS will put an even higher demand on RV follow up (see, e.g.,
a summary in \citealt{exopag2015}).

Doppler spectroscopy also remains the most promising avenue for
detecting Earth-like planet in the Habitable Zone
\citep{1993Icar..101..108K, 2013ApJ...765..131K} in the near
future. Figure~\ref{intro:fig:hz} illustrates the Habitable Zones for
different types of stars and the discovery space that TESS will
access, which does not include the Habitable Zone around Sun-like to
early-M stars due to TESS's short lifespan. The next generation
Doppler spectroscopy, with a RV precision of $<$0.5~m/s, bears great
hope for detecting rocky or even Earth-like planets in the Habitable
Zone. Can we fulfill such a great expectation? The next section
focuses on the art of precise Doppler spectroscopy, on how we achieved
the current precision of 1~m/s today, and on how the field will carry
on and aim for a RV precision of $\sim 10$~cm/s in the coming decade.


%----------------------------------------------------------------
% stellar T vs. planet period, showing RV amplitude curves
% plot provided by Paul Robertson, converted online to eps
\begin{figure}
\centering
\includegraphics[scale=0.55]{introduction/habitable_zone.eps}
\caption{Habitable Zone for stars with various effective temperature,
  highlighted in blue (and red for the extended zone). Some known
  planets in the Habitable Zone of their host stars are plotted, scaled
  by their sizes. Solar system planets are also plotted and scaled by
  size. The yellow and orange dashed vertical lines mark the discovery
  space in terms of planetary periods that TESS is most sensitive
  to. The white curves are equal-RV-amplitude lines, showing the
  semi-amplitude of the RV signals induced by any hypothetical
  Earth-mass planets on each curve. This plot is made by Chester Harman
  and Ravi Kopparapu and kindly provided by Paul Robertson.
\label{intro:fig:hz}}
\end{figure}
%----------------------------------------------------------------



%%%%%%%%%%%%%%%%%%%%%%%%%%%%%%%%%%%%%%%%%%%%%%%%%%%%%%%%%%%%%%%%%%%%%%%%%%%%%%
\section{The Art of Precise Doppler Spectroscopy}

History of precise Doppler spectroscopy. Early work by Campbell and
Walker. First detections with ThAr. Quickly progressed to 3m/s with
iodine. HARPS revoluntionized the field by showing what a cryogenic 
spectrometer can do. Poor NIR...

What is the landscape of RV instruments today. Table of current instruments.

Figure of current precision.

The current best RV precision is around 1~m/s \citep{eprv2015},
attainable via two wavelength calibration methods in the optical band:
ThAr lamp emission line calibration (e.g., ELODIE and HARPS;
\citealt{elodie, harps-s}; $\sim$400-690~nm) and iodine cell
absorption line calibration (e.g., Keck/HIRES and Magellan/PFS;
\citealt{butler1996, 2010SPIE.7735E..53C}; $\sim$500-620~nm). The
major obstacles for achieving a higher RV precision are: stellar
activity induced RV signals, instrumental effects, telluric
contamination, and limitation in data analysis \citep{eprv2015}.

Where is the field heading towards in the near future. Fischer et
al. review, Plavchan NASA white paper. 

Table of Future instruments.

Future RV missions: MINERVA \cite{minerva}, HPF
\cite{2012SPIE.8446E..1SM}, WIYN-NEID, etc.


%%%%%%%%%%%%%%%%%%%%%%%%%%%%%%%%%%%%%%%%%%%%%%%%%%%%%%%%%%%%%%%%%%%%%%%%%%%%%%
\section{Precise Doppler Spectroscopy with Iodine Cells as
  Calibrators} 

Much history in Paul Butler's personal account.

Our work mostly deals with \het\ and \keck, who happen to only have
iodine-calibrated precise RV instruments so far. Unfortunately this
will remain true for a while in the near future too for all large
telescopes. Getting the most out of the iodine calibration method is
therefore meaningful. 

Iodine method will remain relevant in the future too,
because iodine is a relatively cheap way for getting precise RVs,
which will be on high demand in the near future once TESS is
launched. Study why and why not HET works will inform us about other
iodine-calibrated instruments and current and future fiber-fed
instruments. Study how we can improve Keck and generalize its methodology
to other telescopes is important.

The scope of this work mostly deals with data analysis, with the hope
to diagnose problems and inform hardware builders, and also to improve
the data analysis technique in order to measure RVs more precisely.

introduction on \het\ and \keck\ will be in their chapters.