The first discovery of an extra-solar planet (exoplanet) around a
main-sequence star, 51 Peg $b$, discovered using Doppler spectroscopy,
opened up the field of exoplanets. For more than a decade, the
dominant way for finding exoplanets was using precise Doppler
spectroscopy to measure the radial velocity (RV) changes of
stars. Today, precise Doppler spectroscopy is still crucial for the
discovery and characterization of exoplanets, and it has a great
chance for finding the first rocky exoplanet in the Habitable Zone of
its host star. However, such endeavor requires an exquisite precision
of 10-50 cm/s while the current state of the art is 1 m/s.

This thesis set out to improve the RV precision of two precise Doppler
spectrometers on two 10-meter class telescopes: \het\ and \keck. Both
of these spectrometers use iodine cells as their wavelength
calibration sources, and their spectral data are being analyzed via
forward modeling to estimate stellar RVs. Neither \het\ or \keck\
deliver an RV precision at the photon-limited level, meaning that
there are additional RV systematic errors caused by instrumental
changes or errors in the data analysis. \het\ has an RV precision of
3-5 m/s, while \keck\ has about 1-2 m/s. I have found that the leading
cause behind \het's ``under-performance'' in comparison to \keck\ is
temperature changes of the iodine gas cell (and thus an inaccurate
iodine reference spectrum). Another reason is the insufficient
modeling of the \het\ instrumental profile. While \keck\ does not
suffer from these problems, it also has several RV systematic error
sources of considerable sizes. The work in this thesis has revealed
that the errors in \keck's stellar reference spectrum add about 1 m/s
to the error budget and are the major drivers behind the spurious RV
signal at the period of a sidereal year and its harmonics. Telluric
contamination and errors caused by the spectral fitting algorithm also
contribute on the level of 20-50 cm/s. The strategies proposed and
tested in this thesis will improve the RV precision of \het\ and
\keck, including their decade worth of archival data.

This thesis also documents my work on characterizing exoplanet orbits
using RV data and the discovery of HD 37605$c$. It concludes with a
summary of major findings and an outline of future plans to use future
precise Doppler spectrometers to move towards the goal of 10 cm/s and
detecting Earth 2.0.


