\chapter{Characterization of Exoplanet Systems Using Radial
  Velocities}

% boottran and TERMS and Katherina's long-period planet paper.

%---------------------------------------------------------------------------
\section{Introduction and Background}

Extracting exoplanet signals from RV data is hard in many ways. First,
it can be hard to distinguish stellar activity signals from planetary
signals, and one of the examples is the famous case of the GJ 581
system ZZZ cite Mayor, Vogt, Hatze, Robertson ZZZ. Second, even with
the definitive knowledge that the RV signal is dominated by planets,
it can still be challenging if the planetary system is dynamically
active, e.g., for the case of the 55 Cnc system ZZZ cite Ben's paper
ZZZ, and characterizing planetary orbits is a numerically and
computationally challenging problem ZZZ cite Ben's RUNDMC
paper. Third, even if the star is RV quiet and the planetary system is
dynamically quiet, and all orbits can be described by simple Keplerian
orbits, several challenges remain for this ``parameter fitting" or
optimization problem with complicated model selection scenarios: the
number of planets in the system can be hard to pin down ZZZ cite the
APF/HARPS 6 planet system ZZZ; some orbits may not be well constrained
and thus raise ambiguities (e.g., circular orbits or eccentric; ZZZ
cite eccentric hot jupiter papers); it can be computationally
demanding (especially if Bayesian frame is employed).

In 2012, there was very few published codes on performing simple
Keplerian fit using RV data under the context of exoplanet detection
(i.e., handling multi-planets, data from multiple telescopes, etc.;
ZZZ cite Greg Laughlin's systemic ZZZ who else?), especially ones that
were easy to use. The \rvlin\ package by \cite{rvlin} addresses
the problem of simple Keplerian orbital fitting using least-$\chi^2$
algorithm and exploiting the linear parameters (ZZZ which ones) to
speed up the convergence. It handles multi-planet systems and RV data
from multiple telescopes, is very easy to use (simple input
requirements and easy commands; written in IDL), and its typical time
for convergence is within seconds. It is fairly popular and has a
large user group beyond the exoplanet community (ZZZ citations for
RVLIN package).

However, \rvlin\ does not provide estimates for uncertainties on
the best-fit parameters. This becomes a much desirable feature
especially for the Transit Ephemeris Refinement and Monitoring Survey
(TERMS) project \citep{Kane2009}. TERMS follows up RV detected
exoplanets with moderate separation from their bright, nearby host
stars (semi-major axis $>$ few hundreths of an AU) to search for
transiting ``warm" planets (as opposed to the Hot Jupiters), which are
unique and important targets for atmospheric characterization. TERMS
uses \rvlin\ for orbital parameter estimates, but only having the
best-fit Keplerian parameters will not suffice for transit prediction
and transit observation planning. A transit search can only be
feasible if the transit window is well constrained by the RV
data. Otherwise, more RVs need to be collected. As many of the TERMS
targets were reported in the literature a while ago, the predicted
transit ephemerides might have ``drifted" from the true ones as the
predictive power of old RV data faded as time went by. Therefore,
estimating uncertainties on transit ephemerides becomes crucial for
the project.

With the purpose of calculating transit ephemerides and their
uncertainties for TERMS, and also to supplement \rvlin\ with a
tool to estimate uncertainties, I constructed the \boottran\ package,
which uses bootstrapping to calculate uncertainties for the Keplerian
parameters estimated by \rvlin. The statistical justification and
algorithm of \boottran\ is described in the next section. The final
section of this chapter describes my contributing work on
characterizing exoplanetary systems using \boottran\ in several
publications, mostly as the outcome of the TERMS project. The next
chapter, Chapter~\ref{chap:planets}, descirbes the planetary system
around the star HD~37605, which is another example of application of
\rvlin\ and \boottran.



%---------------------------------------------------------------------------
\section{BOOTTRAN: Uncertainties for Orbital Parameters Estimated
  Using Radial Velocities}\label{boottran:sec:boottran}

{\it
  The following texts are originally published in the appendix of
  \cite{wang2012} in {\it ApJ}, and copy right belongs to IOP
  Publishing. They are used in this thesis with permission (with
  minor modification to fit in this chapter). 
}

We have constructed a package, \boottran, to calculate uncertainties
for Keplerian orbital parameter estimates\footnote{Through out this
  thesis, we refer to the ``{\it estimates} of the parameters'' (as
  distinguished from the ``true parameters'', which are not known and
  can only be estimated) simply as the ``parameters''.} and transit
mid-time $T_c$ via bootstrapping
\citep{1981,davison1997bootstrap}. \boottran\ is designed to calculate
error bars for transit ephemerides and the Keplerian orbital fit
parameters output by the RVLIN package\citep{rvlin}, but can also be a
stand-alone package. The two packages, \rvlin\ and \boottran, are
publicly available at http://exoplanets.org/code/ and the
Astrophysical Source Code Library (ASCL.net). Thanks to the simple
concept of bootstrapping, it is computationally very time-efficient
and easy to use.

The basic idea of bootstrap is to resample based on original data
to create bootstrap samples (multiple data replicates); then for
each bootstrap sample, derive orbital parameters or transit parameters
through orbital fitting and calculation. The ensemble of parameters
obtained in this way yields the approximate sampling distribution for
each estimated parameter. The standard deviation of this sampling
distribution is the standard error for the estimate.

We caution the readers here that there are regimes in which the
``approximate sampling distribution" (a frequentist's concept) is not
an estimate of the posterior probability distribution (a Bayesian
concept), and there are regimes (e.g., when limited sampling affects
the shape of the $\chi^2$ surface) where there are qualitative
differences and the bootstrap method dramatically underestimates
uncertainties \citep[e.g., long-period planets when the observations
  are not yet sufficient to pin down the orbital
  period;][]{Ford2005,Bender2012}. In situations with sufficient RV
data, good phase coverage, a sufficient time span of observations and
a good orbital fit, bootstrap often gives a useful estimate of the
parameter uncertainties. For the data considered in
Chapter~\ref{chap:planets} \citep{wang2012}, it was not obvious that
the bootstrap uncertainty estimate would be accurate, as the time span
of observations is only slightly longer than the orbital period of
planet $c$. Nevertheless, we find good agreement between the
uncertainty estimates derived from bootstrap and MCMC calculations in
\cite{wang2012}.

The radial velocity data are denoted as $\lbrace \vec{t},\ \vec{v},\ \vec{\sigma}
\rbrace$, where each $t_i$, $v_i$, $\sigma_i$ represents radial
velocity $v_i$ observed at time (BJD) $t_i$ with velocity uncertainty
$\sigma_i$. Extreme outliers should be rejected in order to preserve the
validity of our bootstrap algorithm. We first derive our estimates for
the true orbital parameters from the original RV data via orbital fitting,
using the RVLIN package: \beq \vec{\beta}
= \mu(\vec{t},\ \vec{v},\ \vec{\sigma}), \eeq where $\vec{\beta}$ is
the best fitted orbital parameters\footnote{As described in \S
  \ref{sec:fit}, this includes the $P$, $T_p$, $K$, $e$, and $\omega$
  for each planet, as well as $\gamma$, $\mbox{d}v/\mbox{d}t$ (if
  applicable), and velocity offsets between instruments/telescopes (if
  applicable) for the system.}. From $\vec{\beta}$, we derive $\lbrace
\vec{t},\ \vec{v}_{best}(\vec{\beta}) \rbrace$, the best-fit model
(here $\vec{t}$ are treated as predictors and thus fixed). Then we can
begin resampling to create bootstrap samples.

Our resampling plan is model-based resampling, where we draw from the
residuals against the best-fit model. For data that come from the
same instrument or telescope, in which case no instrumental offset
needs to be taken into account, we simply draw from all residuals,
$\lbrace \vec{v}-\vec{v}_{best} \rbrace$, with equal probability for
each $(v_i - v_{best,i})$. This new ensemble of residuals, denoted as
${\vec{r^*}}$, is then added to the best-fit model $\vec{v}_{best}$ to
create one bootstrap sample, $\vec{v^*}$ \footnote{We simply use the
  raw residual instead of any form of modified residual, because the
  RV data for any single instrument or telescope are usually close
  enough to homoscedasticity.}. Associated with ${\vec{r^*}}$, the
uncertainties $\vec{\sigma}$ are also re-assigned to $\vec{v^*}$ --
that is, if $v_j - v_{best,j}$ is drawn as $r_k$ and added to $v_k$ to
generate $v^*_k$, then the uncertainty for $v^*_k$ is set to be
$\sigma_j$.

For data that come from multiple instruments or multiple
telescopes, we incorporate our model-based resampling plan to include
stratified sampling. In this case, although data from each instrument
or telescope are close to homoscedastic, the entire set of data are
usually highly heteroscedastic due to stratification in
instrument/telescope radial velocity precision. Therefore, the
resampling process is done by breaking down the data into different
groups, $\lbrace \vec{v_1},\ \vec{v_2},\ \ldots \rbrace$, according to
instrument and/or telescope, and then resample within each
subgroup of data with the algorithm described in last paragraph. The bootstrap
sample is then $\vec{v^*} = \lbrace \vec{v^*}_1,\ \vec{v^*}_2,\ \ldots
\rbrace$.

To construct the approximate sampling distribution of the orbital
parameter estimates $\vec{\beta}$, we compute \beq \vec{\beta^*} =
\mu(\vec{t},\ \vec{v^*},\ \vec{\sigma^*}) \eeq for each bootstrap
sample, $\lbrace \vec{t},\ \vec{v^*},\ \vec{\sigma^*} \rbrace$. The
sampling distribution for each orbital parameter estimate $\beta_i$
can be constructed from the multiple sets of $\vec{\beta^*}$
calculated from multiple bootstrap samples
($\vec{\beta^*}^{(1)},\ \vec{\beta^*}^{(2)},\ \ldots$ from
$\vec{v^*}^{(1)},\ \vec{v^*}^{(2)},\ \ldots$). The standard errors for
$\vec{\beta}$ are simply the standard deviations of the sampling
distributions\footnote{The standard deviation of a sampling
  distribution is estimated in a robust way using the IDL function
  {\it robust\_sigma}, which is written by H. Fruedenreich based on
  the principles of robust estimation outlined in \cite{1983ured.book.....H}.}.

The sampling distribution of the estimated transit mid-time, $T_c$, is
calculated likewise. Here $T_c$ is the transit time for a certain
planet of interest in the system, and is usually specified to be the first
transit after a designated time $T$.  However, the situation is
complicated by the periodic nature of $T_c$. Our approach is to first
calculate, based on the original RV data, $T_{c0}$, the estimated
mid-time of the first transit after time $T_0$ (an arbitrary time
within the RV observation time window of $[\min(\vec{t}),\ \max(\vec{t})]$; $T_{c0}$ 
is also within this window).
Then
\beq
T_c = N\cdot P + T_{c0},
\eeq
where $P$ is the best-estimated period for this planet of interest,
and $N$ is the smallest integer that is larger than $(T - T_{c0})/P$.
Next we compute $T_{c0}^*$ for each bootstrap sample $\lbrace
\vec{t},\ \vec{v^*},\ \vec{\sigma^*} \rbrace$. Given that within the
time window of radial velocity observations
($[\min(\vec{t}),\ \max(\vec{t})]$), the phase of the planet should be
known well enough, it is fair to assume that $T_{c0}$ is an unbiased
estimator of the true transit mid-time. Therefore we assert that
$T_{c0}^*$ has to be well constrained and within the range of
$[T_{c0}-P^*/2,\ T_{c0}+P^*/2]$, where $P^*$ is the period estimated from
this bootstrap sample. If not, then we subtract or add multiple
$P^*$'s until $T_{c0}^*$ falls within the range. Then naturally
\beq
T_c^* = N\cdot P^* + T_{c0}^*.
\eeq
The ensemble of $T_c^*$'s gives the sampling distribution of $T_c$
and its standard error. Note that $T_c^*$ is not necessarily within
the rage of $[T_{c}-P/2,\ T_{c}+P/2]$.

Provided with the stellar mass $M_\star$ and its uncertainty, we
calculate, for each planet in the system, the standard errors for the
semi-major axis $a$ and the {\it minimum mass} of the planet $M_{\rm
  p,min}$ (denoted as $\msini$ in the main text as commonly seen in
literature, but this is a somewhat imprecise notation). As the first
step, the mass function is calculated for the best-fit $\vec{\beta}$
and each bootstrap sample $\vec{\beta^*}$,
\beq
f(P,K,e) = \frac{P K^3
  (1-e)^{3/2}}{2\pi G} = \frac{(M_p\cdot \sin i)^3}{(M_\star+M_p)^2}.
\eeq
The sampling distribution of $f(P,K,e)$ then gives the standard error
of the mass function. The minimum mass of the planet $M_{\rm p,min}$ 
is then calculated by assuming $\sin{i}=1$ and solving for $M_p$.
ZZZ Is this still true?! Standard error of $M_{\rm p,min}$ is derived through simple
propagation of error, as the covariance between $M_\star$ and
$f(P,K,e)$ is probably negligible. ZZZ

For the semi-major axis $a$,
\beq
a^3 = \frac{P^2 G
  (M_\star+M_p)}{4\pi^2} \approx \frac{P^2 G (M_\star + M_{\rm p,min})}{4\pi^2}.
\eeq
The standard error of $P^2$ is calculated from its bootstrap sampling
distribution, and via simple propagation of error we obtain the
standard error of $a$ (neglecting covariance between $P^2$, $M_{\rm
  p,min}$, and $M_\star$).


%---------------------------------------------------------------------------
\section{Characterization of Planetary Systems Using BOOTTRAN}

Who cited our paper?

Mostly, TERMS. What is TERMS. What are the published systems with me
as coauthor doing this type of work.

What did I do for Ketherina's paper?
