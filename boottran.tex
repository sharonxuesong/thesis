\chapter{Characterization of Exoplanet Systems Using Radial
  Velocities}

% boottran and TERMS and Katherina's long-period planet paper.

%---------------------------------------------------------------------------
\section{BOOTTRAN: Uncertainties for Orbital Parameters Estimated
  Using Radial Velocities}\label{boottran:sec:boottran}

The uncertainties listed for the orbital parameter
estimates\footnote{Through out the paper and sometimes in this
  Appendix, we refer to the ``{\it estimates} of the parameters'' (as
  distinguished from the ``true parameters'', which are not known and
  can only be estimated) simply as the ``parameters''.} and transit
mid-time $T_c$ are calculated via bootstrapping
\citep{1981,davison1997bootstrap} using the package BOOTTRAN, which we
have made publicly available (see \S~\ref{sec:fit}). It is designed to
calculate error bars for transit ephemerides and the Keplerian
orbital fit parameters output by the RVLIN
package\citep{2009ApJS..182..205W}, but can also be a stand-alone
package. Thanks to the simple concept of bootstrapping, it is
computationally very time-efficient and easy to use.

The basic idea of bootstrap is to resample based on original data
to create bootstrap samples (multiple data replicates); then for
each bootstrap sample, derive orbital parameters or transit parameters
through orbital fitting and calculation. The ensemble of parameters
obtained in this way yields the approximate sampling distribution for
each estimated parameter. The standard deviation of this sampling
distribution is the standard error for the estimate.

We caution the readers here that there are regimes in which the
``approximate sampling distribution" (a frequentist's concept) is not
an estimate of the posterior probability distribution (a Bayesian
concept), and there are regimes (e.g., when limited sampling affects
the shape of the $\chi^2$ surface) where there are qualitative
differences and the bootstrap method dramatically underestimates
uncertainties \citep[e.g., long-period planets when the observations
  are not yet sufficient to pin down the orbital
  period;][]{Ford2005,Bender2012}. In situations with sufficient RV
data, good phase coverage, a sufficient time span of observations and
a good orbital fit, bootstrap often gives a useful estimate of the
parameter uncertainties. For the data considered in this paper, it
was not obvious that the bootstrap uncertainty estimate would be
accurate, as the time span of observations is only slightly longer than
the orbital period of planet $c$. Nevertheless, we find good agreement
between the uncertainty estimates derived from bootstrap and MCMC
calculations.

The radial velocity data are denoted as $\lbrace \vec{t},\ \vec{v},\ \vec{\sigma}
\rbrace$, where each $t_i$, $v_i$, $\sigma_i$ represents radial
velocity $v_i$ observed at time (BJD) $t_i$ with velocity uncertainty
$\sigma_i$. Extreme outliers should be rejected in order to preserve the
validity of our bootstrap algorithm. We first derive our estimates for
the true orbital parameters from the original RV data via orbital fitting,
using the RVLIN package \citep{2009ApJS..182..205W}: \beq \vec{\beta}
= \mu(\vec{t},\ \vec{v},\ \vec{\sigma}), \eeq where $\vec{\beta}$ is
the best fitted orbital parameters\footnote{As described in \S
  \ref{sec:fit}, this includes the $P$, $T_p$, $K$, $e$, and $\omega$
  for each planet, as well as $\gamma$, $\mbox{d}v/\mbox{d}t$ (if
  applicable), and velocity offsets between instruments/telescopes (if
  applicable) for the system.}. From $\vec{\beta}$, we derive $\lbrace
\vec{t},\ \vec{v}_{best}(\vec{\beta}) \rbrace$, the best-fit model
(here $\vec{t}$ are treated as predictors and thus fixed). Then we can
begin resampling to create bootstrap samples.

Our resampling plan is model-based resampling, where we draw from the
residuals against the best-fit model. For data that come from the
same instrument or telescope, in which case no instrumental offset
needs to be taken into account, we simply draw from all residuals,
$\lbrace \vec{v}-\vec{v}_{best} \rbrace$, with equal probability for
each $(v_i - v_{best,i})$. This new ensemble of residuals, denoted as
${\vec{r^*}}$, is then added to the best-fit model $\vec{v}_{best}$ to
create one bootstrap sample, $\vec{v^*}$ \footnote{We simply use the
  raw residual instead of any form of modified residual, because the
  RV data for any single instrument or telescope are usually close
  enough to homoscedasticity.}. Associated with ${\vec{r^*}}$, the
uncertainties $\vec{\sigma}$ are also re-assigned to $\vec{v^*}$ --
that is, if $v_j - v_{best,j}$ is drawn as $r_k$ and added to $v_k$ to
generate $v^*_k$, then the uncertainty for $v^*_k$ is set to be
$\sigma_j$.

For data that come from multiple instruments or multiple
telescopes, we incorporate our model-based resampling plan to include
stratified sampling. In this case, although data from each instrument
or telescope are close to homoscedastic, the entire set of data are
usually highly heteroscedastic due to stratification in
instrument/telescope radial velocity precision. Therefore, the
resampling process is done by breaking down the data into different
groups, $\lbrace \vec{v_1},\ \vec{v_2},\ \ldots \rbrace$, according to
instrument and/or telescope, and then resample within each
subgroup of data with the algorithm described in last paragraph. The bootstrap
sample is then $\vec{v^*} = \lbrace \vec{v^*}_1,\ \vec{v^*}_2,\ \ldots
\rbrace$.

To construct the approximate sampling distribution of the orbital
parameter estimates $\vec{\beta}$, we compute \beq \vec{\beta^*} =
\mu(\vec{t},\ \vec{v^*},\ \vec{\sigma^*}) \eeq for each bootstrap
sample, $\lbrace \vec{t},\ \vec{v^*},\ \vec{\sigma^*} \rbrace$. The
sampling distribution for each orbital parameter estimate $\beta_i$
can be constructed from the multiple sets of $\vec{\beta^*}$
calculated from multiple bootstrap samples
($\vec{\beta^*}^{(1)},\ \vec{\beta^*}^{(2)},\ \ldots$ from
$\vec{v^*}^{(1)},\ \vec{v^*}^{(2)},\ \ldots$). The standard errors for
$\vec{\beta}$ are simply the standard deviations of the sampling
distributions\footnote{The standard deviation of a sampling
  distribution is estimated in a robust way using the IDL function
  {\it robust\_sigma}, which is written by H. Fruedenreich based on
  the principles of robust estimation outlined in \cite{1983ured.book.....H}.}.

The sampling distribution of the estimated transit mid-time, $T_c$, is
calculated likewise. Here $T_c$ is the transit time for a certain
planet of interest in the system, and is usually specified to be the first
transit after a designated time $T$.  However, the situation is
complicated by the periodic nature of $T_c$. Our approach is to first
calculate, based on the original RV data, $T_{c0}$, the estimated
mid-time of the first transit after time $T_0$ (an arbitrary time
within the RV observation time window of $[\min(\vec{t}),\ \max(\vec{t})]$; $T_{c0}$ 
is also within this window).
Then
\beq
T_c = N\cdot P + T_{c0},
\eeq
where $P$ is the best-estimated period for this planet of interest,
and $N$ is the smallest integer that is larger than $(T - T_{c0})/P$.
Next we compute $T_{c0}^*$ for each bootstrap sample $\lbrace
\vec{t},\ \vec{v^*},\ \vec{\sigma^*} \rbrace$. Given that within the
time window of radial velocity observations
($[\min(\vec{t}),\ \max(\vec{t})]$), the phase of the planet should be
known well enough, it is fair to assume that $T_{c0}$ is an unbiased
estimator of the true transit mid-time. Therefore we assert that
$T_{c0}^*$ has to be well constrained and within the range of
$[T_{c0}-P^*/2,\ T_{c0}+P^*/2]$, where $P^*$ is the period estimated from
this bootstrap sample. If not, then we subtract or add multiple
$P^*$'s until $T_{c0}^*$ falls within the range. Then naturally
\beq
T_c^* = N\cdot P^* + T_{c0}^*.
\eeq
The ensemble of $T_c^*$'s gives the sampling distribution of $T_c$
and its standard error. Note that $T_c^*$ is not necessarily within
the rage of $[T_{c}-P/2,\ T_{c}+P/2]$.

Provided with the stellar mass $M_\star$ and its uncertainty, we
calculate, for each planet in the system, the standard errors for the
semi-major axis $a$ and the {\it minimum mass} of the planet $M_{\rm
  p,min}$ (denoted as $\msini$ in the main text as commonly seen in
literature, but this is a somewhat imprecise notation). As the first
step, the mass function is calculated for the best-fit $\vec{\beta}$
and each bootstrap sample $\vec{\beta^*}$,
\beq
f(P,K,e) = \frac{P K^3
  (1-e)^{3/2}}{2\pi G} = \frac{(M_p\cdot \sin i)^3}{(M_\star+M_p)^2}.
\eeq
The sampling distribution of $f(P,K,e)$ then gives the standard error
of the mass function. The minimum mass of the planet $M_{\rm p,min}$ 
is then calculated by assuming $\sin{i}=1$ and solving for $M_p$.
Standard error of $M_{\rm p,min}$ is derived through simple
propagation of error, as the covariance between $M_\star$ and
$f(P,K,e)$ is probably negligible.

For the semi-major axis $a$,
\beq
a^3 = \frac{P^2 G
  (M_\star+M_p)}{4\pi^2} \approx \frac{P^2 G (M_\star + M_{\rm p,min})}{4\pi^2}.
\eeq
The standard error of $P^2$ is calculated from its bootstrap sampling
distribution, and via simple propagation of error we obtain the
standard error of $a$ (neglecting covariance between $P^2$, $M_{\rm
  p,min}$, and $M_\star$).


%---------------------------------------------------------------------------
\section{Characterization of Planetary Systems Using BOOTTRAN}

Who cited our paper?

Mostly, TERMS. What is TERMS. What are the published systems with me
as coauthor doing this type of work.

What did I do for Ketherina's paper?
