\chapter{The Carlifornia Planet Survey Doppler Code}\label{chap:doppler}

This chapter contains a brief documentation describing the algorithm
and structure of the California Planet Survey (CPS) Doppler code,
which extracts RVs from iodine-calibrated stellar spectra. As of March
2016, no documentation in published or unpublished form existed for
this widely used code, although \cite{butler1996} describes the basics
for the technique of iodine-calibrated precise RV, and some CPS
publications contain description for certain elements of the code
\citep[e.g.,][]{ 2009ApJ...696...75H, 2011ApJ...726...73H,
  2011ApJS..197...26J}.

% history of the code
Earliest documentation of code indicates 1991, co-created by Paul
Butler and Geoff Marcy. Heavily modified by John Johnson for CPS. Paul
Butler also has a version for LCPS, which he now maintains and also
serves as the pipeline for PFS and APF. Later maintained by Howard
Issacson at Berkeley. Our code is from John Johnson, version
2013. This code has been applied to data taken by Keck/HIRES, AAT,
APF, PFS, Lick/Hamilton, Magellan, HET/HRS (this thesis), and so on.

%%%%%%%%%%%%%%%%%%%%%%%%%%%%%%%%%%%%%%%%%%%%%%%%%%%%%%%%%%%%%%%%%%%%%%%%%%%%%%
% basic algorithm, in mathematical form
\section{Basic Formulae}


%%%%%%%%%%%%%%%%%%%%%%%%%%%%%%%%%%%%%%%%%%%%%%%%%%%%%%%%%%%%%%%%%%%%%%%%%%%%%%
% structure of the code
\section{Code Structure and Work Flow}
