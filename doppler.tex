\chapter{The Carlifornia Planet Survey Doppler Code}\label{chap:doppler}

This chapter contains a brief documentation describing the algorithm
and structure of the California Planet Survey (CPS) Doppler code,
which extracts RVs from iodine-calibrated stellar spectra. As of March
2016, no documentation in published or unpublished form existed for
this widely used code, although \cite{butler1996} describes the basics
for the technique of iodine-calibrated precise RV, and some CPS
publications contain description for certain elements of the code
\citep[e.g.,][]{ 2009ApJ...696...75H, 2011ApJ...726...73H,
  2011ApJS..197...26J}.

% history of the code
Earliest documentation of code indicates 1991, co-created by Paul
Butler and Geoff Marcy. Heavily modified by John Johnson for CPS. Paul
Butler also has a version for LCPS, which he now maintains and also
serves as the pipeline for PFS and APF. Later maintained by Howard
Issacson at Berkeley. This code has been applied to data taken by
Keck/HIRES, AAT, APF, PFS, Lick/Hamilton, Magellan, HET/HRS (this
thesis), and so on. Our code is from John Johnson, version 2013.


%%%%%%%%%%%%%%%%%%%%%%%%%%%%%%%%%%%%%%%%%%%%%%%%%%%%%%%%%%%%%%%%%%%%%%%%%%%%%%
% basic algorithm, in mathematical form
\section{Basic Formulae, Algorithm, and Components}

First, we describe the basic mathematics and algorithm behind RV
extraction from iodine-calibrated stellar spectra using the CPS
code. The overall algorithm is to forward model the stellar spectra
using synthetic or empirically derived model spectra, fitting $N$
parameters, one of which is the Doppler shift, $z$.

The model spectra include\footnote{It can also include a model
  spectrum for a faint secondary star, telluric absorption lines (see
  Chapter~\ref{chap:keck} Section~\ref{keck:sec:telluric}), and so
  on.}: a model spectrum for the iodine absorption lines, $F_{\rm
  I_2}(\lambda)$ and a model spectrum for the star,
$F_{*}(\lambda)$. The goal is to use the model the observed,
extracted, and normalized 1-D spectrum, $F_{\rm obs}(x)$, at any given
pixel position (and spectral order), $x$, using these model spectra
and model parameters. The broadening effect of the spectrograph is
described by the spectral response function, or the spectral point
spread function, or the instrumental profile (IP), which we will refer
to as the IP throughout this thesis and is denoted as
$\curlyp(x)$. Hence,
\beq
F_{\rm obs}(x) = \left[ F_{\rm I_2}(\lambda(x)) \times
F_{*}'(\lambda(x)) \right] \ast \curlyp(x),
\eeq
where $\lambda(x)$ is the wavelength solution for the 1-D spectrum,
and $F_{*}'$ is the red-shifted stellar spectrum defined by
$F_{*}'(\lambda) = F_{*}(\lambda\cdot(1+z))$. 




Ideally, these spectra are the ``ground truth"
spectra, i.e. the intrinsic spectra of the source without Doppler
shift or being broadened by the spectrometer. 



%%%%%%%%%%%%%%%%%%%%%%%%%%%%%%%%%%%%%%%%%%%%%%%%%%%%%%%%%%%%%%%%%%%%%%%%%%%%%%
% structure of the code
\section{Code Structure and Work Flow}

Stellar spectra range from 5000\AA\ to 6200\AA. Chopped into
chunks. All working wavelengths are converted into vacuum. 
