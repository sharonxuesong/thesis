\chapter{Conclusions}
\label{sec:wrapup}

We now know that the relationship between galaxies and their stellar populations is more complex than previously imagined, but the work presented here sheds some light on the interplay between the two.

%interplay between galaxies and their stellar populations.%  No longer simply bound ensambles of isolated stars, galaxies posses complex histories and characters driven as much by circumstance as by anything else.

In the KISS study, I dispel the idea that PAH emission is a simple tracer of star-formation rate.  Mass-specific PAH emission shows very little relationship to mass-specific star-formation rate.  Instead, the efficiency of PAH emission scales with galactic stellar mass.  This may be a proxy for galaxy size or morphology, as some works indicate that PAH emission is strongest along the edges of star-forming complexes.  Regardless of the physical cause, this calibration will aid in the use of PAH emission as an indicator of star-forming activity in a variety of environments because it makes a simple and straight-forward connection between the observable PAH luminosity and star-formation rate.

Far-infrared 24\micron emission is a more direct indicator of star-forming activity. I show in Chapter \ref{ch:kiss} that mass-specific 24\micron emission does correlate with mass-specific star-formation rate.  Because the physical relationship between star formation and the thermal dust emission that dominates the 24\micron band is not as severely complicated by particle destruction due to heating and ultraviolet light, one would expect this calibration to be more straight-forward.

Likewise, the host galaxies of quasars appear to be more complex than anticipated.  I detect star-forming activity throughout the host galaxies, albeit at levels lower than expected.  I find evidence for a narrow-line region in the center of the quasar with the highest Eddington Ratio, \pg{0026+129}.  The finding that star-forming activity accounts for the bulk of the extended line emission in quasar host galaxies confirms the speculation in \cite{Netzer2004} that the narrow-line region size to quasar luminosity relationship proposed in \cite{Bennert} is not robust at the highest luminosities.

The lower-than-expected levels of star-forming activity seen in these quasar hosts, particularly in high Eddington Ratio objects, suggest that many of these objects are transitioning into E+A galaxies, like those described in \cite{Cales}, and have already passed the peak of their star formation.  Because the target quasars were selected to be likely sites of star-forming activity, the finding of lower-than-expected star-formation rates suggests that the AGN quenching may be more efficient than previously suspected.  This conjecture is supported by the presence of shocked gas in the four quasar hosts with the highest Eddington Ratios. From this evidence, quasar host galaxies appear to be particularly violent environments.

Despite lower-than-expected star-formation rates, quasar host galaxies are extreme star-forming galaxies.  In Figure \ref{fig:glue}, stellar mass-specific star formation rates are plotted against stellar masses for the KISS galaxies, the seven quasar host galaxies for which I have infrared data, and 318 galaxies from the The GALEX Ultraviolet Atlas of Nearby Galaxies \citep{GildePaz}. The KISS galaxies match the locus of nearby galaxies fairly well; although the KISS sample is more evenly selected, the KISS galaxies and the nearby galaxies from \cite{GildePaz} have in common that they are normal star-forming objects, so this results is not surprising.

In contrast to the KISS galaxies, five of the seven quasar hosts plotted in Figure \ref{fig:glue} fall at the upper envelope of the distribution. This finding is significant in that the quasar hosts fall within the distribution of normal star-forming galaxies, suggesting that the processes of star formation in quasar hosts are not completely different, and also in that they are on the upper envelope of the distribution, indicating that they are extreme variants of star-forming galaxies. With a star-formation rate of over 25 M$_\odot/{\rm yr}$, \pg{1626+554} may be a truly extraordinary object indeed, but, without infrared continuum photometry, it cannot be appropriately placed in Figure \ref{fig:glue}.

Further work in this area will continue to decode the complexities of star formation in different environments.  While the KISS survey focused on infrared star-formation indicators in AGN-free galaxies, the infrared SEDs of AGN hosts are also of particular interest. Mid- and far-infrared maps of quasar hosts would allow us to access obscured star formation in this violent environment; ALMA or Herschel observations of these or similar objects may be of service.  Using KISS as a reference sample of star-forming galaxies, I can easily put in context any future far-infrared observations of quasar host galaxies. Because the line-ratio diagnostics discussed in Section \ref{sec:lineratios} indicate which portions of the quasar host galaxies are the sites of star-formation and which are irradiated by the central black hole, I can make direct comparisons between far-infrared observations of the star-forming regions of the quasar hosts and the AGN-free KISS galaxies.  As seen in Figure \ref{fig:glue}, the quasar hosts have higher-than-typical mass-specific star-formation rates; it would be interesting to see if their far-infrared emission is also extraordinary.  If so, it would suggest that the surplus star formation is occurring in the centers of dense and heavily enshrouded molecular clouds and, if not, it would suggest that the faster-than-average star formation of quasar hosts is a scaled up version of star formation in typical galaxies.  Either way, that result would be significant to our understanding of the assembly of galaxies.

%Deeper maps of the quasar host galaxies could provide us with insights into their morphology and, possibly, detect more interacting intruder galaxies.  Meanwhile, low-redshift studies such as those presented in this thesis continue to provide insight for interpreting high-redshift findings. % Beyond a doubt, the story that galaxies tell will become more complex and interesting.

Likewise, the metallicities of the KISS galaxies will be a significant asset to my analysis of the quasar host galaxies once I have produced my own line-ratio diagrams with CLOUDY.  Because metallicity spreads the locus of star-forming points in the line-ratio diagram along a track, seen in Figure \ref{fig:bpt} and discussed in Section \ref{sec:lineratios}, it will be possible to roughly estimate the metallicity of star-forming locations in the quasar host galaxies.  I will then compare these metallicity estimates for the quasar host galaxies with metallicities of KISS galaxies with similar masses. Alternatively, I can compare the masses of the quasar host galaxies to those of KISS galaxies with similar metallicities; the metallicity information essentially gives me a second and very complementary way of relating these extreme objects back to normal star-forming galaxies.  Because of its volume-limited collection method, the KISS sample is, by design, an excellent benchmark.  These kinds of investigations will allow me to address questions such as: What kind of galaxies were the quasar hosts before the onset of central black hole accretion?  Or, what kind of galaxies will they mature into when accretion ends?  Putting these quasar host galaxies in context fits into the theme of assembling a chronology of the quasar epoch of a typical galaxy.

The assembly of a chronology of star formation is the ultimate goal of many of the works discussed here, especially my own.  With that, our community can address broader issues about the assembly of galaxies and the star-formation history of the universe.  Research like this also allows us to ask questions about our own Milky Way: Was it ever a quasar host?  What were the stages that it went through?  When we see starburst galaxies, are they analogs to an early Milky Way?  In the work I present here I have investigated a representative sample of typical star-forming galaxies as well as a class of extreme objects; these investigations flesh out, in complementary ways, interesting aspects of the chronology of star-forming galaxies.





\begin{figure}%[T!]
\begin{center}
 \includegraphics[width=14cm,angle=0]{Plots/glue.png}
  \caption{Stellar-mass specific star-formation rates  vs. stellar mass for the quasar host galaxies (green), the KISS galaxies (red), and the The GALEX Ultraviolet Atlas of Nearby Galaxies \citep{GildePaz} (black).  The different morphological types of nearby galaxies are marked with different symbols.  Note that the KISS galaxies fall on essentially the same area of the plot as the nearby galaxies, while the quasar hosts fall at the upper envelope of this area.}
\label{fig:glue}
\end{center}
\end{figure}
